%%%%%%%%%%%%%%%%%%%%%%%%%%%%%%%%%%%%%%%%%%%%%%%%%%%%%%%%%%%%%%%%%%%%%%%%%
%
% File: TrickHLAReqt.tex
%
% Purpose: TrickHLA Product Requirements
%
%%%%%%%%%%%%%%%%%%%%%%%%%%%%%%%%%%%%%%%%%%%%%%%%%%%%%%%%%%%%%%%%%%%%%%%%%

\newcommand\documentHistory{
{\bf Author} & {\bf Date} & {\bf Description} \\ \hline \hline
Edwin Z. Crues & June 2020 & TrickHLA Version 3 \\ \hline
}

% This documentation file's change history includes:
% "REVISED BY name(s)" & "revision DATE" & "revision DESCRIPTION"
%  ------------------     -------------     --------------------

\newcommand\DocumentChangeHistory{
{\bf Revised by} & {\bf Date} & {\bf Description} \\ \hline \hline
}

\documentclass[twoside,11pt,titlepage]{report}

%
% Bring in the AMS math environment
%
\usepackage{amsmath}

%
% Bring in the common page setup
%
\usepackage{trickhlaenv}

%
% Bring in the common math nomenclature
%
\usepackage{trickhlamath}

%
% Bring in the model-specific commands
%
\usepackage{TrickHLA}

%
% Bring in the graphics environment
%
\usepackage{graphicx}

%
% Bring in the hyper ref environment
%
\usepackage[colorlinks]{hyperref}
%  keywords for pdfkeywords are separated by commas
\hypersetup{
   pdftitle={TrickHLA Product Requirements},
   pdfauthor={Edwin Z. Crues \\ and \\ Daniel E. Dexter},
   pdfkeywords={Trick, HLA, TrickHLA, Product Requirements},
   pdfsubject={TrickHLA Product Requirements}}

\begin{document}

%%%%%%%%%%%%%%%%%%%%%%%%%%%%%%%%%%%
% Front matter
%%%%%%%%%%%%%%%%%%%%%%%%%%%%%%%%%%%
\pagenumbering{roman}

\docid{DD.mm.20}
\docrev{1.0}
\date{June 2020}
\modelname{\TrickHLA}
\doctype{Product Requirements}
\author{Edwin Z. Crues \\ and \\ Daniel E. Dexter}
\managers{
  Edwin Z. Crues \\ Project Manager \\
  Michael T. Red \\ Simulation and Graphics Branch Chief (ER7) \\
  Robert O. Ambrose \\ Software, Robotics, and Simulation Division Chief}
\pdfbookmark{Title Page}{titlepage}
\makeTrickhlaenvTitlepage

\pdfbookmark{Abstract}{abstract}
%%%%%%%%%%%%%%%%%%%%%%%%%%%%%%%%%%%%%%%%%%%%%%%%%%%%%%%%%%%%%%%%%%%%%%%%%
%
% Purpose: Abstract for TrickHLA
%
% Author: Edwin Z. Crues - 5/19/2020
%
% Modified:
%
%
%%%%%%%%%%%%%%%%%%%%%%%%%%%%%%%%%%%%%%%%%%%%%%%%%%%%%%%%%%%%%%%%%%%%%%%%%

\begin{abstract}
\TrickHLA\ is a middleware model package that provides an interface
framework for enabling IEEE-1516 High Level Architecture (HLA) capabilities
in simulations developed in the Trick Simulation Environment. \TrickHLA\
allows a developer to concentrate on simulation development without needing
to be an HLA expert. The \TrickHLA\ model is data driven and provides a
simplified API making it relatively easy to take an existing Trick-based
simulation and make it HLA capable.

\end{abstract}


\pdfbookmark{Contents}{contents}
\tableofcontents
\vfill

\pagebreak

%%%%%%%%%%%%%%%%%%%%%%%%%%%%%%%%%%%
% Main Document Body
%%%%%%%%%%%%%%%%%%%%%%%%%%%%%%%%%%%
\pagenumbering{arabic}

%----------------------------------
\chapter{Introduction}\label{sec:intro}
%----------------------------------

%%%%%%%%%%%%%%%%%%%%%%%%%%%%%%%%%%%%%%%%%%%%%%%%%%%%%%%%%%%%%%%%%%%%%%%%%
%
% Purpose: Introduction for TrickHLA
%
% Author: Edwin Z. Crues - 19 May 2020
%
% Modified:
%
%
%%%%%%%%%%%%%%%%%%%%%%%%%%%%%%%%%%%%%%%%%%%%%%%%%%%%%%%%%%%%%%%%%%%%%%%%%

The objective of \TrickHLA\ is to simplify the process of providing simulations
built with the Trick Simulation Environment\cite{Trick:Documentation} with
the ability to participate in distributed executions using the High Level
Architecture (HLA)\cite{IEEE1516:FRAMEWORK}. This allows a simulation developer
to concentrate on the simulation and not have to be an HLA expert.
\TrickHLA\ is data driven and provides a simple API making it relatively easy
to take an existing Trick simulation and make it HLA capable.


\section{Identification of Document}
This document describes the \TrickHLA\
model developed for use in the Trick Simulation Environment.
This document adheres to the documentation standards defined in
NASA Software Engineering Requirements Standard \cite{NASA:SWE}.

\section{Scope of Document}
This document provides information on the requirements for \TrickHLA.

\section{Purpose and Objectives of Document}
The purpose of this document is to define the set of requirements that
the \TrickHLA\ must achieve to be compatible with Federate Inferface
Specification of the IEEE Standard for Modeling and Smulation (M\&S)
High Level Architecture (HLA) \cite{IEEE1516:API}.

\section{Documentation Status and Schedule}
The information in this document is current with the \TrickHLAid\
implementation of the \TrickHLA. Updates will be kept current with
module changes.

\begin{tabular}{||l|l|l|} \hline
\documentHistory
\end{tabular}

\begin{tabular}{||l|l|l|} \hline
\DocumentChangeHistory
\end{tabular}

\section{Document Organization}
This document is organized into the following sections:

\begin{description}

\item[Chapter \ref{sec:intro}: Introduction] -
Identifies this document, defines the scope and purpose, present status,
and provides a description of each major section.

\item[Chapter \ref{sec:docs}: Related Documentation] -
Lists the related documentation that is applicable to this project.

\item[Chapter \ref{sec:reqts}: Requirements] -
Presents the requirements for the \TrickHLA.

\item[Bibliography] -
Informational references associated with this document.

\end{description}

\chapter{Related Documentation}\label{sec:docs}

\section{Parent Documents}
The following documents are parent to this document:

\begin{itemize}
\item{\href{file:TrickHLA.pdf}
           {\em Trick High Level Architecture (\TrickHLA)}}
\cite{trickhlaenv:TrickHLA}
\end{itemize}

\section{Applicable Documents}
The following top level documents are applicable to this document:

\begin{itemize}
\item{\href{file:TrickHLASpec.pdf}
           {\em \TrickHLA\ Product Specification}}
\cite{trickhlaenv:TrickHLASpec}

\item{\href{file:TrickHLAUser.pdf}
           {\em \TrickHLA\ User Guide}}
\cite{trickhlaenv:TrickHLAUser}

\item{\href{file:TrickHLAIVV.pdf}
           {\em \TrickHLA\ Inspection, Verification, and Validation}}
\cite{trickhlaenv:TrickHLAIVV}

\end{itemize}

The following specific documents are applicable to this document:

\begin{itemize}
\item{\em IEEE Standard for Modeling and Simulation (M\&S) High Level
          Architecture (HLA) - Federate Inferface Specification}
\cite{IEEE1516:API}

\item{\em IEEE Standard for Modeling and Simulation (M\&S) High Level
              Architecture (HLA) - Object Model Template (OMT) Specification}
\cite{IEEE1516:OMT}

\end{itemize}

The following additional documents are applicable to this document:

\begin{itemize}
\item{\em Trick Simulation Environment: Installation Guide}
\cite{Trick:Install}

\item{\em Trick Simulation Environment: Tutorial}
\cite{Trick:Tutorial}

\item{\em Trick Simulation Environment: Documentation}
\cite{Trick:Documentation}

\item{\em NASA Software Engineering Requirements}
\cite{NASA:SWE}

\end{itemize}


\chapter{Requirements}\label{sec:reqts}

\section{General Requirements}\label{sec:general_reqts}

This section identifies general requirements for the \TrickHLA.


\requirement{Documentation}
\label{reqt:documentation}
\begin{description}
  \item[Requirement:]\ \newline
    The documentation for the model shall include

    \subrequirement{}
    \label{reqt:reqts_doc}
      Software requirements specification.

    \subrequirement{}
    \label{reqt:design_doc}
      Software, interface, and software version descriptions.

    \subrequirement{}
    \label{reqt:test_doc}
      Software test procedures and results.

    \subrequirement{}
    \label{reqt:user_doc}
      User Guide.

  \item[Rationale:]\ \newline
    The listed items are needed to comply with NASA NPR 7150.2
    as a Class C product.

  \item[Verification:]\ \newline
    Inspection
\end{description}


\requirement{Header File Trick Header}
\label{reqt:h_trick_header}
\begin{description}
  \item[Requirement:]\ \newline
    All header files associated with the model shall have an appropriate
    Trick header. The Trick header for a header file shall include

    \subrequirement{Purpose}\label{reqt:h_trick_header_purpose}
      A brief description of the file.

    \subrequirement{References}\label{reqt:h_trick_header_refs}
      A list of applicable references that describe the model.

    \subrequirement{Assumptions and limitations}
    \label{reqt:h_trick_header_assum}
      A list of the assumptions made in developing the model and
      any limitations on the use of the model.

    \subrequirement{Programmer}\label{reqt:h_trick_header_prog}
      A list of the developers who created or modified the file.

  \item[Rationale:]\ \\[-20pt]
    \begin{itemize}
      \item The presence of a Trick header in a header file
        indicates that Trick should process the file.
      \item Properly documenting the TrickHLA package models
        is a key goal of the TrickHLA verification,
        validation, and documentation task.
      \item Maintaining a version history is good programming
        technique and is mandatory per NPR 7150.2.
    \end{itemize}

  \item[Verification:]\ \newline
    Inspection
\end{description}

\requirement{Trick Comments for Enumerated Types}
\label{reqt:enum_trick_comments}
\begin{description}
  \item[Requirement:]\ \newline
    Each tag defined in a enumeration type in a model header
    file {\em should} have a comment describing the tag that follows
    the tag declaration.

  \item[Rationale:]\ \newline
    Short tag names may not suffice in establishing
    the meaning of the tag.

  \item[Verification:]\ \newline
    Inspection \newline
    Enumerated types that fail to meet this optional requirement
    shall be noted as such in the model verification document.
\end{description}


\requirement{Trick Comments for Data Structures}
\label{reqt:struct_trick_comments}
\begin{description}
  \item[Requirement:]\ \newline
    Each element of a data structure defined in a model header
    file shall have a Trick-compliant comment describing the
    element that follows the element declaration.

  \item[Rationale:]\ \newline
    The element comment is required by Trick.

  \item[Verification:]\ \newline
    Inspection
\end{description}


\requirement{Source File Trick Headers}
\label{reqt:c_trick_header}
\begin{description}
  \item[Requirement:]\ \newline
    Each externally visible function defined in the source files
    associated with the model shall have an appropriate Trick header.
    The Trick header for a function shall include

    \subrequirement{Purpose}\label{reqt:c_trick_header_purpose}
      A brief description of the function.

    \subrequirement{References}\label{reqt:c_trick_header_refs}
      A list of applicable references that describe the function.

    \subrequirement{Assumptions and limitations}
    \label{reqt:c_trick_header_assum}
      A list of the assumptions made in developing the function and
      any limitations on the use of the function.

    \subrequirement{Class}
    \label{reqt:c_trick_header_class}
      The default Trick job classification of the function.

    \subrequirement{Library dependency}
    \label{reqt:c_trick_header_depend}
      A list of the object files upon which the function depends,
      starting with the current file.

    \subrequirement{Programmer}\label{reqt:c_trick_header_prog}
      A list of the developers who created or modified the file.

  \item[Rationale:]\ \\[-20pt]
    \begin{itemize}
      \item The Trick header that precedes a function
        indicates that the function is available for
        use in a simulation {\em S\_define} file.
      \item Properly documenting the TrickHLA package models
        is a key goal of the TrickHLA verification,
        validation, and documentation task.
      \item Maintaining a version history is good programming
        technique and is mandatory per NPR 7150.2.
    \end{itemize}

  \item[Verification:]\ \newline
    Inspection
\end{description}


\requirement{Trick Comments for Function Definitions}
\label{reqt:func_trick_comments}
\begin{description}
  \item[Requirement:]\ \newline
    Each function shall be commented with a Trick-compliant
    set of comments that describe the return value from the
    function and that describe the nature of the arguments
    passed to the function.

  \item[Rationale:]\ \newline
    The function definition comments are required by Trick.

  \item[Verification:]\ \newline
    Inspection
\end{description}


\requirement{HLA Federate Interface}
\label{reqt:hla_federate_interface}
\begin{description}
  \item[Requirement:]\ \newline
    The \TrickHLA\ model shall use the HLA federate interface
    specification defined by IEEE standard 1516.1-2010 \cite{IEEE1516:API}.

  \item[Rationale:]\ \newline
    The \TrickHLA\ model is being designed to use the HLA interfaces
    specified in the IEEE 1516.1-2010 standard \cite{IEEE1516:API}.

  \item[Verification:]\ \newline
    Inspection
\end{description}



\section{Data Requirements}\label{sec:data_reqts}

This section identifies requirements on the data
represented by the \TrickHLA. These as-built requirements are
based on the \TrickHLA\ data definition header files.


\requirement{Primitive Data Types}
\label{reqt:primitive_data_types}
\begin{description}
  \item[Requirement:]\ \newline
    The \TrickHLA\ model shall support the C and C++ primitive
    data types supported by the Trick simulation environment.

    \subrequirement{bool}\label{reqt:primitive_data_types_bool}
      \ \newline
      The {\em bool} Boolean data type for C++ shall be supported.

    \subrequirement{char}\label{reqt:primitive_data_types_char}
      \ \newline
      The {\em char} integer data type for both C and C++
      shall be supported.

    \subrequirement{unsigned char}\label{reqt:primitive_data_types_uchar}
      \ \newline
      The {\em unsigned char} integer data type for both C and C++
      shall be supported.

    \subrequirement{short}\label{reqt:primitive_data_types_short}
      \ \newline
      The {\em short} integer data type for both C and C++
      shall be supported.

    \subrequirement{unsigned short}\label{reqt:primitive_data_types_ushort}
      \ \newline
      The {\em unsigned short} integer data type for both C and C++
      shall be supported.

    \subrequirement{int}\label{reqt:primitive_data_types_int}
      \ \newline
      The {\em int} integer data type for both C and C++
      shall be supported.

    \subrequirement{unsigned int}\label{reqt:primitive_data_types_uint}
      \ \newline
      The {\em unsigned int} integer data type for both C and C++
      shall be supported.

    \subrequirement{long}\label{reqt:primitive_data_types_long}
      \ \newline
      The {\em long} integer data type for both C and C++
      shall be supported.

    \subrequirement{unsigned long}\label{reqt:primitive_data_types_ulong}
      \ \newline
      The {\em unsigned long} integer data type for both C and C++
      shall be supported.

    \subrequirement{long long}\label{reqt:primitive_data_types_llong}
      \ \newline
      The {\em long long} integer data type for both C and C++
      shall be supported.

    \subrequirement{unsigned long long}\label{reqt:primitive_data_types_ullong}
      \ \newline
      The {\em unsigned long long} integer data type for both C and C++
      shall be supported.

    \subrequirement{float}\label{reqt:primitive_data_types_float}
      \ \newline
      The {\em float} floating-point data type for both C and C++
      shall be supported.

    \subrequirement{double}\label{reqt:primitive_data_types_double}
      \ \newline
      The {\em double} floating-point data type for both C and C++
      shall be supported.

  \item[Rationale:]\ \newline
    The C and C++ primitive data types are expected to be
    needed by the various simulation developers.

  \item[Verification:]\ \newline
    Inspection or Test
\end{description}


\requirement{Static Arrays of Primitive Data Types}
\label{reqt:static_arrays_of_primitive_data_types}
\begin{description}
  \item[Requirement:]\ \newline
    The \TrickHLA\ model shall support static arrays of C and C++
    primitive data types.

  \item[Rationale:]\ \newline
    Static arrays of C and C++ primitive data types are expected to be
    needed by the various simulation developers.

  \item[Verification:]\ \newline
    Inspection or Test
\end{description}


\section{Functional Requirements}\label{sec:func_reqts}

This section identifies requirements on the functional capabilities
provided by the \TrickHLA. These as-built requirements are based
on the \TrickHLA\ source files.


\requirement{Data Driven}
\label{reqt:data_driven}
\begin{description}
  \item[Requirement:]\ \newline
    The \TrickHLA\ model shall be data driven so that it can be configured
    in the Trick Simulation Input file.

  \item[Rationale:]\ \newline
    A data driven design will allow the \TrickHLA\ model configuration
    to be modified without needing to recompile the simulation.

  \item[Verification:]\ \newline
    Inspection
\end{description}

\requirement{HLA Big and Little Endian}
\label{reqt:hla_endianess}
\begin{description}
  \item[Requirement:]\ \newline
    The \TrickHLA\ model shall be able to automatically translate
    primitive data types to either Big or Little Endian based on the
    specified Endianess of the corresponding HLA attribute or parameter.

  \item[Rationale:]\ \newline
    The Federation Object Model (FOM) used by HLA defines what data,
    type, Endianess, and encoding to be used to exchanged data
    between federates. The \TrickHLA\ model must be able to translate to
    and from the data types specified in the FOM used for exchanging data
    using HLA.

  \item[Verification:]\ \newline
    Inspection or Test
\end{description}


\requirement{HLA Encoding}
\label{reqt:hla_encoding}
\begin{description}
  \item[Requirement:]\ \newline
    The \TrickHLA\ model shall support the HLAunicodeString,
    HLAASCIIstring, and HLAopaqueData encodings as specified in the
    HLA Object Model Template Specification \cite{IEEE1516:OMT}.

  \item[Rationale:]\ \newline
    Strings may be encoded at the HLA interface using either
    HLAunicodeString, HLAASCIIstring, or HLAopaqueData. The \TrickHLA\
    model uses the HLAunicodeString and HLAASCIIstring for encoding
    strings that need to pass through the HLA interface. The \TrickHLA\
    model uses the HLAopaqueData encoding for encoding a generic array
    of bytes.

  \item[Verification:]\ \newline
    Inspection or Test
\end{description}


\requirement{Time Advancement}
\label{reqt:hla_time_advancement}
\begin{description}
  \item[Requirement:]\ \newline
    The \TrickHLA\ model shall automatically handle time advancement.

  \item[Rationale:]\ \newline
    The HLA standard provides an interface for managing time advancement
    among the federates in the federation and should be coordinated
    with the simulation.

  \item[Verification:]\ \newline
    Inspection or Test
\end{description}


\requirement{Lag Compensation}
\label{reqt:lag_compensation}
\begin{description}
  \item[Requirement:]\ \newline
    The \TrickHLA\ model shall provide an interface to the simulation
    to allow for none, send-side, or receive-side lag compensation.

  \item[Rationale:]\ \newline
    Lag compensation is a technique where the latency between the time
    data is sent a one federate and received at another can be compensated
    for by the simulation builder.

  \item[Verification:]\ \newline
    Inspection or Test
\end{description}


\requirement{Interactions}
\label{reqt:hla_interactions}
\begin{description}
  \item[Requirement:]\ \newline
    The \TrickHLA\ model shall provide an interface to the simulation
    to allow for sending and receiving of interactions as either
    Receive Order or Timestamp Order.

  \item[Rationale:]\ \newline
    Interactions can be either an asynchronous or synchronous
    communication of data between federates. Interactions are expected
    to be needed by the various simulation developers.

  \item[Verification:]\ \newline
    Inspection or Test
\end{description}


\requirement{Ownership Transfer}
\label{reqt:hla_ownership_transfer}
\begin{description}
  \item[Requirement:]\ \newline
    The \TrickHLA\ model shall provide an interface for transfering
    ownership of data attributes between federates.

  \item[Rationale:]\ \newline
    Ownership transfer of data attributes is expected to be
    needed by the various simulation developers.

  \item[Verification:]\ \newline
    Inspection or Test
\end{description}


\requirement{Dynamic Initialization}
\label{reqt:dynamic_initialization}
\begin{description}
  \item[Requirement:]\ \newline
    The \TrickHLA\ model shall provide an interface for dynamically
    initializing the data of a simulation.

  \item[Rationale:]\ \newline
    It is expected that the federates will need to exchange initialization
    data before the simulation starts.

  \item[Verification:]\ \newline
    Inspection or Test
\end{description}


\requirement{Automatic Simulation Startup}
\label{reqt:automatic_sim_startup}
\begin{description}
  \item[Requirement:]\ \newline
    The \TrickHLA\ model shall automatically synchronize the startup
    of a distributed simulation given a list of federates required to
    run the simulation.

  \item[Rationale:]\ \newline
    Ensuring that all the required federates have joined the federation
    and start executing the simulation at the same time is expected to be
    needed by the various simulation developers.

  \item[Verification:]\ \newline
    Inspection or Test
\end{description}


\requirement{Pack / Unpack of Simulation Data}
\label{reqt:pack_unpack}
\begin{description}
  \item[Requirement:]\ \newline
    The \TrickHLA\ model shall provide an interface to allow preprocessing
    (pack) of data before it is sent through HLA interface and post-processing 
    (unpack) the data received from the HLA interface.

  \item[Rationale:]\ \newline
    A simulation developer may need to perform calculations or other work 
    before data is sent to and/or received from the HLA interface.

  \item[Verification:]\ \newline
    Inspection or Test
\end{description}


\requirement{Object Deleted Notification}
\label{reqt:object_deletion}
\begin{description}
  \item[Requirement:]\ \newline
    The \TrickHLA\ model shall provide an interface to allow notification of 
    an object being deleted from the federation.

  \item[Rationale:]\ \newline
    A simulation developer may need to carry out specific actions when a data 
    object is deleted from the federation.

  \item[Verification:]\ \newline
    Inspection or Test
\end{description}


\requirement{Federation Restore}
\label{reqt:federation_restore}
\begin{description}
  \item[Requirement:]\ \newline
    The \TrickHLA\ model shall provide an interface to allow the user to
    programmatically request a federation restore from their trick model.

  \item[Rationale:]\ \newline
    A simulation developer may need to request a federation restore from their
    trick model at any time.
    \newline
    {\tt NOTE}: In the current implementation, the programmatic call to
    restore a federation cannot happen during federation execution; it can only
    occur when the federation is starting up. 

  \item[Verification:]\ \newline
    Inspection or Test
\end{description}


\requirement{Federation Save}
\label{reqt:federation_save}
\begin{description}
  \item[Requirement:]\ \newline
    The \TrickHLA\ model shall provide an interface to allow the user to
    programmatically request a federation save from their trick model.

  \item[Rationale:]\ \newline
    A simulation developer may need to request a federation save from their
    trick model at any time.

  \item[Verification:]\ \newline
    Inspection or Test
\end{description}


\requirement{Conditional sending of attributes}
\label{reqt:conditional_attribute_send}
\begin{description}
  \item[Requirement:]\ \newline
    The \TrickHLA\ model shall provide an interface to allow the user to
    programmatically identify when to send attributes from their trick model.

  \item[Rationale:]\ \newline
    A simulation developer may need send only changed attributes to other
    federates rather then burning up bandwitdh to send the full complement of
    attributes when only a few may have actually changed.

  \item[Verification:]\ \newline
    Inspection or Test
\end{description}


\requirement{Multiple verbosity levels}
\label{reqt:multiple_verbosity_levels}
\begin{description}
  \item[Requirement:]\ \newline
    The \TrickHLA\ model shall provide an interface to allow the user to
    specify multiple levels of verbosity from TrickHLA.

  \item[Rationale:]\ \newline
    A simulation developer may wish to see different levels of information from
    the inner workings of \TrickHLA\ software to aid in debugging simulation
    problems.

  \item[Verification:]\ \newline
    Inspection or Test
\end{description}


%%%%%%%%%%%%%%%%%%%%%%%%%%%%%%%%%%%%%%%%%%%%%%%%%%%%%%%%%%%%%%%%%%%%%%%%%
% Bibliography
%%%%%%%%%%%%%%%%%%%%%%%%%%%%%%%%%%%%%%%%%%%%%%%%%%%%%%%%%%%%%%%%%%%%%%%%%
\newpage
\pdfbookmark{Bibliography}{bibliography}
\bibliography{trickhlaenv,TrickHLA,IEEE1516}
\bibliographystyle{plain}

\end{document}
