\chapter{Object Deleted Notification}
\label{sec:hla-objDel}

\TrickHLA\ provides a mechanism to run user specified code when an object is 
deleted from the federation. This capability allows for user specified 
operation to be performed when an object is deleted.

% ---------------------------------------------------------------------------
\section{What is an {\em ObjectDeleted} class?}

\TrickHLA\ provides a {\tt TrickHLAObjectDeleted} class with a {\tt deleted()} 
method that is used for notification of a deleted object. This is a virtual 
method and must be overridden by a derived class in order to add 
application-specific functionality to the simulation.

% ------------------------------------
\subsection{{\tt TrickHLAObjectDeleted}}

The header file for the {\tt TrickHLAObjectDeleted} class is shown below.

\begin{lstlisting}[caption={{\tt TrickHLAObjectDeleted} class header}]
class TrickHLAObject;
#include "TrickHLA/include/TrickHLAObject.hh"

class TrickHLAObjectDeleted
{
   friend class InputProcessor;
   friend void init_attrTrickHLAObjectDeleted();

  public:
   TrickHLAObjectDeleted() {};
   virtual ~TrickHLAObjectDeleted() {};

   virtual void deleted (     // RETURN: -- None.
      TrickHLAObject * ) {};  // IN: -- Deleted object data.
};
\end{lstlisting}

% ------------------------------------
\subsection{{\tt simplesine\_objectDeleted}}

In order to illustrate the use of the {\tt TrickHLAObjectDeleted} class, we 
subclass it, as shown below.

\begin{lstlisting}[caption={{\tt simplesine\_objectDeleted} class header}]
#include "TrickHLA/include/TrickHLAObjectDeleted.hh"

class simplesine_objectDeleted : public TrickHLAObjectDeleted
{
   friend class InputProcessor;
   friend void init_attrsimplesine_objectDeleted();

  public:
   simplesine_objectDeleted();          // default constructor
   virtual ~simplesine_objectDeleted(); // destructor

   void deleted(            // RETURN: -- None.
        TrickHLAObject * ); // IN:     -- Deleted object.
};
\end{lstlisting}

We give the {\tt deleted()} method something to do, as shown below.

\begin{lstlisting}[caption={ {\tt simplesine\_objectDeleted} code}]
/********************************* TRICK HEADER *******************************
PURPOSE: (simplesine_objectDeleted : Callback class the user writes to do
          something once the object has been deleted from the RTI.)
LIBRARY DEPENDENCY: ((simplesine_objectDeleted.o))
*******************************************************************************/
// System include files.
#include <sstream>

// Trick include files.
#include "sim_services/include/exec_proto.h"

// TrickHLA model include files.

// Model include files.
#include "../include/simplesine_objectDeleted.hh"

using namespace std;

/********************************* TRICK HEADER *******************************
PURPOSE: (simplesine_objectDeleted::simplesine_objectDeleted : Default 
          constructor.)
*******************************************************************************/
simplesine_objectDeleted::simplesine_objectDeleted() // RETURN: -- None.
   : TrickHLAObjectDeleted()
{ }

/********************************* TRICK HEADER *******************************
PURPOSE: (simplesine_objectDeleted::~simplesine_objectDeleted : Destructor.)
*******************************************************************************/
simplesine_objectDeleted::~simplesine_objectDeleted() // RETURN: -- None.
{ }

/********************************* TRICK HEADER *******************************
PURPOSE: (simplesine_objectDeleted::deleted : Callback routine implementation
          to report that this object has been deleted from the federation.)
LIBRARY DEPENDENCY: ((TrickHLAObject.o))
*******************************************************************************/
void simplesine_objectDeleted::deleted( // RETURN: -- None.
   TrickHLAObject * obj)                // IN: -- Deleted object.
{
   ostringstream msg;
   msg << "object '" << obj->get_name()
       << "' resigned from the federation." << endl;
   send_hs( stdout, (char *) msg.str().c_str() );
}
\end{lstlisting}

As you can see, all the {\tt deleted()} method does is just echoes a message 
to the simulation window.

% ---------------------------------------------------------------------------
\section{{\tt S\_define} file}

The {\tt simplesine\_objectDeleted} class is introduced into the simulation via 
the {\tt S\_define} file. There, you would need to add a 
new {\tt simplesine\_objectDeleted} object into each {\tt sim\_object} to which 
you wish to add a callback, like the following:

\begin{verbatim}
simplesine: simplesine_objectDeleted      obj_deleted_callback;
\end{verbatim}

Additionally, you need to add a scheduled job into \TrickHLA\ simulation 
object, like the following:

\begin{verbatim}
(PROPAGATE_TIMESTEP, scheduled) TrickHLA: THLA.manager.process_deleted_objects();
\end{verbatim}

thereby scheduling the \TrickHLA's manager to identify any newly deleted 
objects every data cycle. When an object has been deleted from the federation, 
the manager will trigger, only once, all registered callback 
methods {\tt [deleted()]} (see the next section).


% ---------------------------------------------------------------------------
\section{{\tt input} file}

You need to register the {\tt simplesine\_objectDeleted} object to the THLA 
manager by adding the following lines.

\begin{verbatim}
THLA.manager.objects[0].deleted = &subscriber.obj_deleted_callback;
. . .
THLA.manager.objects[1].deleted = &publisher.obj_deleted_callback;
\end{verbatim}

These lines associate obj\_deleted\_callback as the callback code for the 
subscriber and publisher {\tt sim\_object}s, respectively.

% ---------------------------------------------------------------------------
\section{Output}

The following output sample shows the callback code in action.

\begin{lstlisting}[caption={output displayed on the console}]
...
| |rat|1|0.00|2008/06/06,12:27:10| TrickHLAFedAmb::initialize()
| |rat|1|0.00|2008/06/06,12:27:10| TrickHLAFederate::initialize()
| |rat|1|0.00|2008/06/06,12:27:10| TrickHLAFederate::restart_initialization()
| |rat|1|0.00|2008/06/06,12:27:10| TrickHLAManager::IMSim_initialization_version_1()
| |rat|1|0.00|2008/06/06,12:27:10| TrickHLAManager::setup_all_ref_attributes()
| |rat|1|0.00|2008/06/06,12:27:10| TrickHLAManager::setup_object_ref_attributes()
...
| |rat|1|15.00|2008/06/06,12:27:26| object 'P-side-Federate.Test' resigned from the federation.
...
SIMULATION TERMINATED IN
  PROCESS: 1
  JOB/ROUTINE: 21/sim_services/mains/master.c
DIAGNOSTIC:
Simulation reached input termination time.

LAST JOB CALLED: THLA.THLA.federate.time_advance_request()
              TOTAL OVERRUNS:            0
PERCENTAGE REALTIME OVERRUNS:        0.000%


       SIMULATION START TIME:        0.000
        SIMULATION STOP TIME:       15.000
     SIMULATION ELAPSED TIME:       15.000
         ACTUAL ELAPSED TIME:       15.000
        ACTUAL CPU TIME USED:        4.847
    SIMULATION / ACTUAL TIME:        1.000
       SIMULATION / CPU TIME:        3.095
  ACTUAL INITIALIZATION TIME:        0.000
     INITIALIZATION CPU TIME:        0.674
*** DYNAMIC MEMORY USAGE ***
     CURRENT ALLOCATION SIZE:  1411651
       NUM OF CURRENT ALLOCS:      888
         MAX ALLOCATION SIZE:  1413600
           MAX NUM OF ALLOCS:      894
       TOTAL ALLOCATION SIZE:  1726197
         TOTAL NUM OF ALLOCS:     5991
\end{lstlisting}
