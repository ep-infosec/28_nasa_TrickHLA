\chapter{Timeline}
\label{sec:timeline}

\TrickHLA\ provides a mechanism for the user to specify the scenario timeline
for the simulation. \TrickHLA\ needs access to the scenario timeline in order
to coordinate freezing (pausing) the federation. The scenario timeline is the
only timeline that can be counted on to be synchronized between all the federates.

% ---------------------------------------------------------------------------
\section{What is the {\em TrickHLATimeline} class?}

\TrickHLA\ provides a {\tt TrickHLATimeline} class with a {\tt get\_time()} 
method that is used for getting the current simulation scenario time. This is
a virtual method and must be overridden by a derived class in order to add 
application-specific functionality to the simulation. If a secnario timeline
is not specified by the user then \TrickHLA\ will use the Trick simulation time
as the default scenario timeline, which is only valid if all Federates are using
Trick and start with the same simulation time.

% ---------------------------------------------------------------------------
\subsection{{\tt TrickHLATimeline}}

The header file for the {\tt TrickHLATimeline} class is shown below.

\begin{lstlisting}[caption={{\tt TrickHLATimeline} class header}]
class TrickHLATimeline
{
   friend class InputProcessor;
   friend void init_attrTrickHLATimeline();

  public:
   TrickHLATimeline();
   virtual ~TrickHLATimeline();
  private:
   TrickHLATimeline(const TrickHLATimeline & rhs);
   TrickHLATimeline & operator=(const TrickHLATimeline & rhs);

  public:
   virtual double get_time(); // Returns a time in seconds, typically
                              // Terrestrial Time (TT) for the Scenario Timeline.
};
\end{lstlisting}

% ---------------------------------------------------------------------------
\subsection{{\tt TrickHLASimTimeline}}

In order to illustrate the use of the {\tt TrickHLATimeline} class, we  subclass
it, as shown below.

\begin{lstlisting}[caption={{\tt TrickHLASimTimeline} class header}]
#include "TrickHLA/include/TrickHLATimeline.hh"

class TrickHLASimTimeline : public TrickHLATimeline
{
   friend class InputProcessor;
   friend void init_attrTrickHLASimTimeline();

  public:
   TrickHLASimTimeline();          // default constructor
   virtual ~TrickHLASimTimeline(); // destructor

   virtual double get_time(); // RETURN: s Current simulation time in seconds to represent the scenario time.
};
\end{lstlisting}

We give the {\tt get\_time()} method something to do, as shown below.

\begin{lstlisting}[caption={ {\tt TrickHLASimTimeline} code}]
/********************************* TRICK HEADER *******************************
PURPOSE: (TrickHLASimTimeline : This class represents the simulation timeline.)
LIBRARY DEPENDENCY: ((TrickHLASimTimeline.o))
*******************************************************************************/
// Trick include files.
#if TRICK_VER >= 10
#  include "sim_services/Executive/include/Executive.hh"
#  include "sim_services/Executive/include/exec_proto.h"
#else
   // Trick 07
#  include "sim_services/include/executive.h"
#  include "sim_services/include/exec_proto.h"
#endif

// TrickHLA include files.
#include "TrickHLA/include/TrickHLASimTimeline.hh"

/********************************* TRICK HEADER *******************************
PURPOSE: (TrickHLASimTimeline::TrickHLASimTimeline : Default constructor.)
*******************************************************************************/
TrickHLASimTimeline::TrickHLASimTimeline() // RETURN: -- None.
{ }

/********************************* TRICK HEADER *******************************
PURPOSE: (TrickHLASimTimeline::~TrickHLASimTimeline : Destructor.)
*******************************************************************************/
TrickHLASimTimeline::~TrickHLASimTimeline() // RETURN: -- None.
{ }

/********************************* TRICK HEADER *******************************
PURPOSE: (TrickHLASimTimeline::get_time() : Get the current simulation time.)
LIBRARY DEPENDENCY: ((TrickHLATimeline.o)(TrickHLASimTimeline.o))
*******************************************************************************/
double TrickHLASimTimeline::get_time() // RETURN: -- Current simulation time in seconds to represent the scenario time.
{
#if TRICK_VER >= 10
   return exec_get_sim_time();
#else
   return exec_get_exec()->out.time;
#endif
}
\end{lstlisting}

In this example, all the {\tt get\_time()} method does is just return the
Trick simulation time.

% ---------------------------------------------------------------------------
\section{{\tt S\_define} file}

The {\tt TrickHLASimTimeline} class is introduced into the simulation via 
the {\tt S\_define} file. There, you would need to add a 
new {\tt TrickHLASimTimeline} object into one simulation object and in this
example we add it to the THLA\_INIT simulation object like the following:

\begin{verbatim}
TrickHLA: TrickHLASimTimeline  sim_timeline;
\end{verbatim}

\TrickHLA\ will call the get\_time() function when it needs to get the current
scenario time.

% ---------------------------------------------------------------------------
\section{{\tt input} file}

You need to register the {\tt TrickHLATimeline} object with the THLA 
federate by adding the following lines.

\begin{verbatim}
THLA.federate.scenario_timeline = &THLA_INIT.sim_timeline
\end{verbatim}

The simulation scenario timeline is specified by the THLA\_INIT.sim\_timeline
implementation.

