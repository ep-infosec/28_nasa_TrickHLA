%%%%%%%%%%%%%%%%%%%%%%%%%%%%%%%%%%%%%%%%%%%%%%%%%%%%%%%%%%%%%%%%%%%%%%%%%%%%%%
%%% TrickHLA SpaceFOM FESFA Template
%%%%%%%%%%%%%%%%%%%%%%%%%%%%%%%%%%%%%%%%%%%%%%%%%%%%%%%%%%%%%%%%%%%%%%%%%%%%%%
%%%
%%% Purpose:
%%%   This document provides a LaTeX based implementation of the SISO
%%%   SpaceFOM Federation Execution Specific Federation Agreement (FESFA).
%%%   This is a template document that is intended to provide a starting
%%%   point for anyone that needs to create SpaceFOM-compliant FESFA.
%%%
%%%   The document contains editorial notes, in red.  These notes can be
%%%   revealed or hidden using the LaTeX 'multiaudience' package.  This
%%%   document also contains example place holder text, marked with the
%%%   \example{} text entries.  These examples are intended to be replaced
%%%   with normal and appropriate text entries.
%%%
%%% Revision History:
%%%   January 2021: Edwin Z. Crues: Initial version based on SISO templates.
%%%
%%%%%%%%%%%%%%%%%%%%%%%%%%%%%%%%%%%%%%%%%%%%%%%%%%%%%%%%%%%%%%%%%%%%%%%%%%%%%%

%%%%%%%%%%%%%%%%%%%%%%%%%%%%%%
% Primary document style
\documentclass[11pt,english,letterpaper]{article}
\setlength{\parskip}{\baselineskip}
\setlength{\parindent}{0pt}
%%%%%%%%%%%%%%%%%%%%%%%%%%%%%%

%%%%%%%%%%%%%%%%%%%%%%%%%%%%%%
% Required packages
\usepackage[english]{babel}
%\usepackage{helvet}
\usepackage{textcomp}
\usepackage{enumitem}
\usepackage{xcolor}
\usepackage{multiaudience}
\usepackage{fancyhdr} % needed for header and footer
\usepackage{lastpage} % needed for page numbering 1 of 2, 2 of 2, etc...
\usepackage[colorlinks]{hyperref}
\hypersetup{
   colorlinks=true,
   linkcolor=blue,
   filecolor=green,
   citecolor=blue,
   urlcolor=red,
   plainpages=false,
   pdfpagelabels=true
}
\usepackage{biblatex}
\usepackage{tabularx} % Better tables.
%%%%%%%%%%%%%%%%%%%%%%%%%%%%%%

%%%%%%%%%%%%%%%%%%%%%%%%%%%%%%
% Page spacing options
\topmargin 0in 
\headheight 0.25in
\headsep 12pt  
\textheight 9in 
\textwidth 6.5in
\oddsidemargin 0in
\evensidemargin 0in
%%%%%%%%%%%%%%%%%%%%%%%%%%%%%%

%%%%%%%%%%%%%%%%%%%%%%%%%%%%%%%%%%%%%%%%%%%%%%%%%%%%%%%%%%%%%%%%%%%%%%%%%%%%%%
% End of preamble
%%%%%%%%%%%%%%%%%%%%%%%%%%%%%%%%%%%%%%%%%%%%%%%%%%%%%%%%%%%%%%%%%%%%%%%%%%%%%%
%%%%%%%%%%%%%%%%%%%%%%%%%%%%%%%%%%%%%%%%%%%%%%%%%%%%%%%%%%%%%%%%%%%%%%%%%%%%%%

\SetNewAudience{editorial}
\DefCurrentAudience{editorial} % Uncomment this to see editorial text.

\newcommand{\example}[1]{{\textcolor{blue}{\textit{#1}}}}

\addbibresource{IEEE1516.bib}
\addbibresource{SpaceFOM_FESFA_Template.bib}

%%%%%%%%%%%%%%%%%%%%%%%%%%%%%%%%%%%%%%%%%%%%%%%%%%%%%%%%%%%%%%%%%%%%%%%%%%%%%%
%%%%%%%%%%%%%%%%%%%%%%%%%%%%%%%%%%%%%%%%%%%%%%%%%%%%%%%%%%%%%%%%%%%%%%%%%%%%%%
% Starting Document Content
%%%%%%%%%%%%%%%%%%%%%%%%%%%%%%%%%%%%%%%%%%%%%%%%%%%%%%%%%%%%%%%%%%%%%%%%%%%%%%
\begin{document}

\begin{center}
\begin{LARGE}
{\bfseries Space Reference Federation Object Model\\
(SpaceFOM)\\
Federation Execution Specific Federation Agreement\\
(FESFA)}\\
\vspace{10pt}
\end{LARGE}
{\normalsize for the}\\
\vspace{10pt}
\begin{LARGE}
{\bfseries \example{\textlangle{}Federation Execution Title\textrangle{}}}
\end{LARGE}
\end{center}

\begin{shownto}{editorial}
{\color{red} The information that appears in red text is intended
to be editorial and directive content.  It is not intended to appear in the
finished document.  This text can be hidden by commenting out or deleting the

\texttt{\textbackslash DefCurrentAudience\{editorial\}}

line in this document's \texttt{*.tex} file.  The \example{blue italic text}
represents example text and is intended to be replaced with the Federattion
Execution specific text.  When complete, the document should not have any red
or {\color{blue}blue} text, other than hyprelinked text.

The Space Reference Federation Object Model (SpaceFOM) Federation Execution
Specific Federation Agreement (FESFA) is a document that provides specific
configuration data necessary to achieve interoperability based on the SpaceFOM.
Several rules in the SpaceFOM put requirements on what data need to be recorded
in the FESFA. This template establishes the standard format and content so that
all SpaceFOM FESFA products contain the same basic information and have the
same basic look.}
\end{shownto}


%%%%%%%%%%%%%%%%%%%%%%%%%%%%%%%%%%%%%%%%%%%%%%%%%%%%%%%%%%%%%%%%%%%%%%%%%%%%%%
\section*{Purpose}

\example{ This section of the FESFA template will provide the general purpose
and description of this specific SpaceFOM-compliant federation execution.
This should include intended scenarios and other information that describes
the nature of the federates participating in a federation execution compliant
with this FESFA.\cite{IEEE1516:FRAMEWORK,IEEE1516:API,IEEE1516:OMT}}


%%%%%%%%%%%%%%%%%%%%%%%%%%%%%%%%%%%%%%%%%%%%%%%%%%%%%%%%%%%%%%%%%%%%%%%%%%%%%%
\section*{Identification}

\begin{shownto}{editorial}
{\color{red} This section of the FESFA template provides the general
identifying information associates with the federation execution.

General name identifying the federation execution, this should match the
\textlangle{}Federation Execution Title\textrangle{} in the title above but not
necessarily the “HLA Federation Execution Name” below.}
\end{shownto}

\textbf{Federation Execution Title: } \underline{\example{Your Federation Execution Title Here}}

\begin{shownto}{editorial}
{\color{red} Information pertaining to the principal point of contact for this FESFA.}
\end{shownto}

\textbf{Point of Contact: }

\hspace{0.25in}
\begin{tabularx}{\textwidth}{lX}
Name:    & \underline{\example{POC Name}} \\
Phone:   & \underline{\example{POC Phone Number}} \\
Email:   & \underline{\example{POC Email Address}} \\
Address: & \underline{\example{POC Phydical Address}}
\end{tabularx}

\begin{shownto}{editorial}
{\color{red} Real world time frame (calendar dates) for proposed federation
executions, not to be confused with federation execution scenario dates.}
\end{shownto}

\textbf{Planned Execution Time Frame: }

\hspace{0.25in}
\begin{tabular}{ll}
From: & \underline{\example{Earliest Date}} \\
To:   & \underline{\example{Latest Date}}
\end{tabular}

\begin{shownto}{editorial}
{\color{red} HLA federation execution name, not necessarily the identification
name from above.}
\end{shownto}

\textbf{HLA Federation Execution Name: } \underline{\example{The Federation Execution Name}}


%%%%%%%%%%%%%%%%%%%%%%%%%%%%%%%%%%%%%%%%%%%%%%%%%%%%%%%%%%%%%%%%%%%%%%%%%%%%%%
\section*{Federation Composition}

\begin{shownto}{editorial}
{\color{red} This section of the FESFA template provides the identifying
information associates with the composition of the federation execution.

Information on the federate providing the role of the Master Federate for this
federation execution.}
\end{shownto}

\textbf{Master Federate: }
\underline{\example{Federate Name}}

\begin{shownto}{editorial}
{\color{red} Information on the federate providing the role of the Pacing
Federate for this federation execution.}
\end{shownto}

\textbf{Pacing Federate: }
\underline{\example{Federate Name}}

\begin{shownto}{editorial}
{\color{red} Information on the federate providing the role of the Root
Reference Frame Publisher (RRFP) federate for this federation execution.}
\end{shownto}

\textbf{Root Reference Frame Publisher (RRFP): }
\underline{\example{Federate Name}}

\begin{shownto}{editorial}
{\color{red} List the names and descriptions of any additional required
federates for this federation execution. Add additional lines as needed.}
\end{shownto}

\textbf{Additional required federates: }

\begin{tabularx}{\textwidth}{|l|X|} \hline
\textbf{Name} & \textbf{Description} \\ \hline
\example{Federate Name 1} &
\example{Potentially lengthy multiline description.  This could be really long
long long long long long long long.} \\ \hline
\example{Federate Name 2} &
\example{Potentially lengthy multiline description.} \\ \hline
\example{etc.} &  \\ \hline
\end{tabularx}


%%%%%%%%%%%%%%%%%%%%%%%%%%%%%%%%%%%%%%%%%%%%%%%%%%%%%%%%%%%%%%%%%%%%%%%%%%%%%%
\section*{Time Management}

\begin{shownto}{editorial}
{\color{red} This section of the FESFA template provides the general time
management information associates with the federation execution. All
participating time managed federates will need this information. Some of this
information is published by the Master Federate in the Execution Control Object
(ExCO).

The starting date for the federation execution federation scenario time (FST\textsubscript{0}).
This can be given as a calendar date and time but will ultimately have to be
converted into the Terrestrial Time (TT) scale in Truncated Julian Date (TJD)
format, in seconds.}
\end{shownto}

\textbf{Epoch: } \underline{\example{Starting date and time.}} (TT scale in TJD format)

\begin{shownto}{editorial}
{\color{red} The federation execution's nominal HLA Logical Time (HLT) step in
microseconds.}
\end{shownto}

\textbf{Federation HLT step: } \underline{\example{numerical time step}} (microseconds)

\begin{shownto}{editorial}
{\color{red} The federation execution's Least Common Time Step (LCTS) in
microseconds. This is the least common value of all the federate time steps for
the time regulating federates in a federation execution.}
\end{shownto}

\textbf{Federation LCTS: }  \underline{\example{numerical time step}} (microseconds)

\begin{shownto}{editorial}
{\color{red} Identify the supported time management type for this federation execution.}
\end{shownto}

\textbf{Supported Time Management Types: }

\hspace{0.25in}
\begin{tabular}{ll}
No Pacing:                                & \underline{\example{(yes/no)}} \\
Scaled Pacing:                            & \underline{\example{(yes/no)}} \\
Real-time Pacing with Unlimited Overruns: & \underline{\example{(yes/no)}} \\
Real-time Pacing with Limited Overruns:   & \underline{\example{(yes/no)}} \\
Strict/Conservative Real-time Pacing:     & \underline{\example{(yes/no)}} \\
\end{tabular}

\begin{shownto}{editorial}
{\color{red} If any of the real-time pacing options are supported, then include
a section that describes limitations and how those overruns are handled.}
\end{shownto}

\textbf{Overrun handling: } \underline{\example{Description of how overruns
are handled.}}

\begin{shownto}{editorial}
{\color{red} Indication of the existence of Central Timing Equipment (CTE) to
control the federation execution time advance for real-time hardware-in-the-loop
(HwITL) simulations. This is a yes or no question.}
\end{shownto}

\textbf{CTE federates exists: } \underline{\example{(yes/no)}}

\begin{shownto}{editorial}
{\color{red} References to any document(s) that describes the implementation
and configuration details for any CTE. Add additional document references as
necessary. Just mark (N/A) if no CTE is used.}
\end{shownto}

\textbf{CTE specification document(s): }
\begin{enumerate}
\item \example{CTE reference document 1.}
\item \example{CTE reference document 2.}
\end{enumerate}


%%%%%%%%%%%%%%%%%%%%%%%%%%%%%%%%%%%%%%%%%%%%%%%%%%%%%%%%%%%%%%%%%%%%%%%%%%%%%%
\section*{Reference Frames}

\begin{shownto}{editorial}
{\color{red} This section of the FESFA template provides the names and
descriptions of the principal reference frames published during a federation
execution. It should be sufficient to understand the principal topology of the
federation execution's reference frame tree.

The name and brief description of the reference frame that represents the
common base (root) reference frame for the federation execution's reference
frame tree.}
\end{shownto}

\textbf{Root Reference Frame: }

\begin{tabularx}{\textwidth}{lX}
Name & Description \\
\example{Root Frame Name} &
\example{A short description of the root reference frame.  If not apparent from
the name and short description, a reference should be provided.\cite{vallado2001}}
\end{tabularx}

\begin{shownto}{editorial}
{\color{red} The name, parent, and brief description of any additional reference
frame that play an important role in the federation execution's reference frame
tree. Any reference frames published by or subscribed to by required federates
should be listed here. This list should represent the union of all reference
frames listed in the FCDs of the required federates. It may also include
reference frames of other potential federates. Add additional lines as
necessary.}
\end{shownto}

\textbf{Additional Reference Frames: }

\begin{tabularx}{\textwidth}{|l|l|X|} \hline
Name & Parent & Description \\ \hline
\example{EarthCentericInertial} &
\example{Root Frame Name} &
\example{Description of the Earth centered inertial reference frame.} \\ \hline
\example{EarthCentericFixed} &
\example{EarthCenteredInertial} &
\example{Description of the Earth centered planet fixed frame.} \\ \hline
\example{MoonCentricInertial} &
\example{Root Frame Name} &
\example{Description of the Moon centered inertial reference frame.} \\ \hline
\end{tabularx}

%%%%%%%%%%%%%%%%%%%%%%%%%%%%%%%%%%%%%%%%%%%%%%%%%%%%%%%%%%%%%%%%%%%%%%%%%%%%%%
\section*{Object Management}

\begin{shownto}{editorial}
{\color{red} This section of the FESFA template provides the general object
management information associated with the federation execution. Most
participating federates will need this information.

List the type strings associated with any PhysicalEntity object's `type'
attribute used in this federation execution. List both the string (tag) values
and a description of each tag.}
\end{shownto}

\textbf{Physical Entity Object Type Strings: }

\begin{tabularx}{\textwidth}{|l|X|} \hline
Type String (Tag) & Description \\ \hline
\example{Tag 1} & \example{The description of the tag.} \\ \hline
\example{Tag 2} & \example{The description of the tag.} \\ \hline
\example{etc.} & \\ \hline
\end{tabularx}

\begin{shownto}{editorial}
{\color{red} List the status strings associated with any PhysicalEntity object's
`status' attribute used in this federation execution. List both the string (tag)
values and a description of each tag.}
\end{shownto}

\textbf{Physical Entity Object Status Strings: }

\begin{tabularx}{\textwidth}{|l|X|} \hline
Status String (Tag) & Description \\ \hline
\example{String 1} & \example{The description of the string.} \\ \hline
\example{String 2} & \example{The description of the string.} \\ \hline
\example{etc.} &  \\ \hline
\end{tabularx}

\begin{shownto}{editorial}
{\color{red} A brief description of the canonical naming convention used to
distinguish PhysicalInterface object instances from one another. Add additional
lines as necessary.}
\end{shownto}

\textbf{Physical Interface Instance Naming Convention: }

\example{Provide a detailed explanation of the \texttt{PhysicalInterface}
instance naming convention used to uniquely identify the interfaces to
\texttt{PhysicalEntities}.}

\begin{shownto}{editorial}
{\color{red} The name, type, and brief description of any key object instances
that play an important role in the federation execution. Add additional lines
as necessary.}
\end{shownto}

\textbf{Key Object Instances: }

\begin{tabularx}{\textwidth}{|l|l|X|} \hline
Name & Object Class & Description \\ \hline
\example{Instance Name 1} & \example{\texttt{ObjectType}} &
\example{The description of the string.} \\ \hline
\example{Instance Name 2} & \example{\texttt{ObjectType}} &
\example{The description of the string.} \\ \hline
\example{etc.} & \\ \hline
\end{tabularx}

\begin{shownto}{editorial}
{\color{red} List the name and description of any additional FOM modules needed
by this federation execution. This should be the union of all FOM modules
listed in the FCDs of the required federates. It may also include other FOM
modules of other potential federates. Add additional lines as necessary.}
\end{shownto}

\textbf{Additional FOM Modules: }

\begin{tabularx}{\textwidth}{|l|X|} \hline
FOM Module Name & Description \\ \hline
\example{Module Name 1} & \example{The description of the FOM module.} \\ \hline
\example{Module Name 2} & \example{The description of the FOM module.} \\ \hline
\example{etc.} &  \\ \hline
\end{tabularx}


%%%%%%%%%%%%%%%%%%%%%%%%%%%%%%%%%%%%%%%%%%%%%%%%%%%%%%%%%%%%%%%%%%%%%%%%%%%%%%
\section*{Initialization}

\begin{shownto}{editorial}
{\color{red} This section of the FESFA template provides the information
associates with the initialization policy and approach use in the federation
execution. It specifically focuses on the details of any multiphase
initialization. All Early Joiner federates participating in the multiphase
initialization process will need this information.

Indication for the use of multiphase initialization (MPI). This is a yes or no
question.}
\end{shownto}

\textbf{MPI Used: } \underline{\example{(yes/no)}}

\begin{shownto}{editorial}
{\color{red} The MPI specification can consist of an inline description of the
MPI approach. Alternately, list references to any document(s) that describes
the implementation and configuration details for any MPI used in the startup of
the federation execution. Add additional document references as necessary. Just
mark (N/A) if no MPI is used.}
\end{shownto}

\textbf{MPI Specification: }
\begin{enumerate}
\item \example{MPI reference document 1.}
\item \example{MPI reference document 2.}
\end{enumerate}


%%%%%%%%%%%%%%%%%%%%%%%%%%%%%%%%%%%%%%%%%%%%%%%%%%%%%%%%%%%%%%%%%%%%%%%%%%%%%%
\section*{Additional Technical Information}

\begin{shownto}{editorial}
{\color{red} This section of the FESFA template provides any additional
technical information needed by federates participating in the federation
execution. This section may be marked (N/A) or omitted if there is no
additional technical information.

Specify any non-standard switches settings required to configure the RTI for
this federation execution. For instance, this is where the behavior of the
Auto-Provide switch would be documented if enabled. Add additional lines as
necessary. Just mark (N/A) or omit if none.}
\end{shownto}

\textbf{Non-standard Switches Settings: }

\begin{tabularx}{\textwidth}{|l|l|X|} \hline
\textbf{Switch} & \textbf{Value} & \textbf{Description} \\ \hline
\example{Switch} & \example{\texttt{Value}} &
\example{The description of the switch setting.} \\ \hline
\example{Switch} & \example{\texttt{Value}} &
\example{The description of the switch setting.} \\ \hline
\example{etc.} &  & \\ \hline
\end{tabularx}

\begin{shownto}{editorial}
{\color{red} List or describe any additional data sources and/or databases
required to support this federation execution. Add additional lines as
necessary. Just mark (N/A) or omit if no additional data sources are needed.}
\end{shownto}

\textbf{Additional Common Data and/or Databases: }

\begin{tabularx}{\textwidth}{|l|X|} \hline
\textbf{Data Source} & \textbf{Data Description} \\ \hline
\example{Source 1} & 
\example{The description of the data.} \\ \hline
\example{Source 2} & 
\example{The description of the data.} \\ \hline
\example{etc.} &  \\ \hline
\end{tabularx}

\begin{shownto}{editorial}
{\color{red} References to any additional technical document. Add additional
document references as necessary. Just mark (N/A) or omit if none.}
\end{shownto}

\textbf{Additional Technical Documents: }
\begin{enumerate}
\item \example{Technical reference document 1.}
\item \example{Technical reference document 2.}
\end{enumerate}

\begin{shownto}{editorial}
{\color{red} The next section is automatically generated by \LaTeX using the
BibLaTeX package and Biber program.  A bibliography will be generated from
\LaTeX citations that occur in the document.}
\end{shownto}


%%%%%%%%%%%%%%%%%%%%%%%%%%%%%%%%%%%%%%%%%%%%%%%%%%%%%%%%%%%%%%%%%%%%%%%%%%%%%%
\printbibliography

\end{document}
%%%%%%%%%%%%%%%%%%%%%%%%%%%%%%%%%%%%%%%%%%%%%%%%%%%%%%%%%%%%%%%%%%%%%%%%%%%%%%
% End Document Content
%%%%%%%%%%%%%%%%%%%%%%%%%%%%%%%%%%%%%%%%%%%%%%%%%%%%%%%%%%%%%%%%%%%%%%%%%%%%%%
%%%%%%%%%%%%%%%%%%%%%%%%%%%%%%%%%%%%%%%%%%%%%%%%%%%%%%%%%%%%%%%%%%%%%%%%%%%%%%

