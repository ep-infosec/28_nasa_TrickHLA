\section{The {\tt simplesine} Model}
\label{sec:simplesine-model}

In the following chapters,
we use a simple sine wave model, {\tt simplesine}, to illustrate \TrickHLA\ in action.
Understanding the model and how it is used in a Trick \sdefine file
is important for those chapters.
This section introduces \simplesine with that in mind.

% -----------------------------------------------------------------------
\subsection{Description}

The system modeled by \simplesine is an undamped harmonic oscillator,
the dynamics of which are governed by the differential equation
\begin{equation}
\ddot{x} + w^2 x = 0,
\label{eq:EOM}
\end{equation}
which has an analytic solution of the form
\begin{subequations}
\label{eq:analytic-equations}
\begin{align}
x(t)       &= A        \sin {(\omega t + \phi)}, \\
\dot{x}(t) &= A \omega \cos {(\omega t + \phi)}.
\end{align}
\end{subequations}

The relevant model data are
the dynamic {\em state}, $(x, \dot{x})$ and
the constant system {\em parameters}, $(A, \phi, \omega)$,
where the parameters are specified as inputs
and the state is calculated dynamically as simulation outputs.\footnote{
In this model, the initial state cannot be specified as inputs explicitly
but rather through the parameters $A$ and $\phi$.
}

The \simplesine model has functions that may be used to calculate the state
analytically based on equations~\ref{eq:analytic-equations}.
It can also propagate the state approximately
based on numerical integration of the differential equations.
The integration involves the calculation of the derivative of
the 2-vector $\boldsymbol{z}$ defined as

\begin{equation}
\label{eq:differential-equations}
  \boldsymbol{z}(t)
  \equiv \left\{
            \begin{array}{c}
              x(t)\\ \dot{x}(t)
            \end{array}
     \right\} \\
  = \left\{
            \begin{array}{r}
              A \sin{(\omega t + \phi)} \\ A \omega \cos{(\omega t + \phi)}
            \end{array}
     \right\}
\end{equation}

So that

\begin{equation}
\label{eq:derivative-equations}
  \dot{\boldsymbol{z}}(t)
  = \left\{
            \begin{array}{c}
              \dot{x}(t)\\ \ddot{x}(t)
            \end{array}
     \right\} \\
  = \left\{
            \begin{array}{r}
              \dot{x}(t)\\ - \omega^2 x(t)
            \end{array}
     \right\} \\
  = \left\{
            \begin{array}{r}
                    A w \cos{(\omega t + \phi)} \\
                    - A w^2 \omega \sin{(\omega t + \phi)}
            \end{array}
     \right\}
\end{equation}

% -------------------------------------
\subsection{Model}

The source code for the \simplesine model is organized into three directories:
{\tt data}, {\tt include} and {\tt src}.
The files in these directories are discussed below.

% ----------
\subsubsection{Include Files}

The \simplesine C/C++ include files are in directory {\tt simplesine/include}.
A list of the files is shown below.

{
\begin{center}
\scriptsize
\begin{tabular}{|l|l|}
\hline
{\em filename} \\
\hline
\hline
{\tt simplesine.h} \\
\hline
{\tt simplesine\_InteractionHandler.h} \\
\hline
{\tt simplesine\_LagCompensator.h} \\
\hline
{\tt simplesine\_Packing.h} \\
\hline
{\tt simplesine\_proto.h} \\
\hline
\end{tabular}
\end{center}
}

\paragraph{\tt simplesine.h}
This file declares the fundamental \simplesine data structures.
There is a single data structure that in turn holds state- and parameter-related
data structures.
In the simulations that follow, the {\tt simplesine\_T} structure will be
frequently declared as a {\tt sim\_object} in the \sdefine files
when the simulations need to model a sine wave.

The {\tt simplesine\_T} structure consists of a {\em state} substructure,
which holds $x$ and $\dot{x}$
and a parameters substructure which holds the constant {\em parameters,}
$A$, $\phi$ and $\omega$.
In this model, only the parameters
may be set from the input processor.
The state may only be calculated by a Trick job.
The purpose of this is to ensure that the state and parameters are never
initialized to inconsistent values.\footnote{
  Of course, this requires that developers remember to explicitly
  call {\tt simplesine\_calc()} as an initialization job for every
  \simplesine sim variable.
}

The complete file is shown in Appendix~\ref{sec:simplesine-h}.

\paragraph{\tt simplesine\_proto.h}
This file declares the C functions exported by the \simplesine model.
These functions may be used in an \sdefine file as Trick jobs.
Functions of particular interest are
\begin{itemize}
  \item{\tt simplesine\_calc()}, which calculates the state, $(x, \dot{x})$
    according to equations~\ref{eq:analytic-equations},\footnote{
      Since only the \simplesine parameters have default values,
      this job may be used as an initialization job to initialize
      the state from the parameters.
    }
  \item{\tt simplesine\_deriv()} and {\tt simplesine\_integ()}, which
    are used to numerically integrate equations~\ref{eq:differential-equations}
    using the standard Trick integration scheme,
  \item{\tt simplesine\_copyXXX()}, several routines which copy \simplesine
    data from one data structure to another, and
  \item{\tt simplesine\_calcError()}, which calculates the error between
one \simplesine state and the true values based on equations~\ref{eq:analytic-equations}.
\end{itemize}


The complete file is shown in Appendix~\ref{sec:simplesine-proto-h}.

\paragraph{\tt simplesine\_InteractionHandler.hh}
This file declares the \simplesine C++ class which acts as a
\TrickHLA\ {\em interaction handler}.
It is a subclass of {\tt TrickHLAInteractionHandler},
which is the \TrickHLA\ class which defines how interactions
are sent and received.

The complete file is shown in Appendix~\ref{sec:simplesine-InteractionHandler-hh}.

\paragraph{\tt simplesine\_LagCompensator.hh}
This file declares the \simplesine C++ class which acts as a
\TrickHLA\ {\em lag compensator}.
It is a subclass of {\tt TrickHLALagCompensator},
which is the \TrickHLA\ class which defines how federates may compensate
for HLA-time lags created as a result of sending data to remote federates
and transfering ownership between federates.

The complete file is shown in Appendix~\ref{sec:simplesine-LagCompensator-hh}.

\paragraph{\tt simplesine\_Packing.hh}
This file declares the \simplesine C++ class which may be optionally
used by developers to {\em pack} outbound data prior to sending via HLA
and {\em upack} is upon receipt from HLA.
It is a subclass of {\tt TrickHLAPacking},
which has {\tt pack()} and {\tt unpack()} virtual methods that implement
the application-specific packing and unpacking logic.

The complete file is shown in Appendix~\ref{sec:simplesine-Packing-hh}.


% ----------
\subsubsection{Source Files}

The \simplesine C/C++ source files are in the directory {\tt simplesine/src}.

The implementation of the \simplesine functions and classes is in
{\tt .c} and {\tt .cpp} files located in the {\tt simplesine/src} directory.
The C functions are mainly implemented one function per file\footnote{
  The copy functions
  ({\tt simplesine\_copyParams()},
  {\tt simplesine\_copyParams()}, and
  {\tt simplesine\_copyParams()})
  are located in a single file, {\tt simplesine\_copy.c}.
},
and the C++ classes are implemented one class per file.
The file names are shown in the table below:

{
\begin{center}
\scriptsize
\begin{tabular}{|l|l|}
\hline
{\em filename} & {\em implements what?} \\
\hline
\hline
{\tt simplesine\_calc.c} & {\tt simplesine\_copy()} \\
\hline
{\tt simplesine\_calcError.c} & {\tt simplesine\_calcError()} \\
\hline
{\tt simplesine\_compensate.c} & {\tt simplesine\_compensate()} \\
\hline
{\tt simplesine\_copy.c} & the {\tt simplesine\_copyXXX()} functions \\
\hline
{\tt simplesine\_deriv.c} & {\tt simplesine\_deriv()} \\
\hline
{\tt simplesine\_integ.c} & {\tt simplesine\_integ()} \\
\hline
{\tt simplesine\_propagate.c} & {\tt simplesine\_propagate()} \\
\hline
\hline
{\tt simplesine\_InteractionHandler.cpp} & the \TrickHLA\ interaction handler class \\
\hline
{\tt simplesine\_LagCompensator.cpp} & the \TrickHLA\ lag compensator class \\
\hline
{\tt simplesine\_Packing.cpp} & the \TrickHLA\ packing/unpacking class \\
\hline
\end{tabular}
\end{center}
}


% ----------
\subsubsection{Data Files}

This {\tt simplesine/data} directory
consists Trick input files for default \simplesine data.

Depending on how you build your \sdefine file, these files may be used
as ``fallback'' initializations for your data.
In the simulations that follow, the actual initial values
will often override the defaults in these files.\footnote{
  Sine the \simplesine state data are output-only,
  they cannot be set directly from the Trick input processor.
  Consequently, there is no default data file for
  {\tt simplesine\_state\_T}.
}

The files are shown below.

{
\begin{center}
\scriptsize
\begin{tabular}{|l|l|}
\hline
{\em filename} & {\em description} \\
\hline
\hline
{\tt integ.d}
  &
  Default numerical integration parameters.
  \\
\hline
{\tt simplesine.d}
  &
  Default \simplesine parameters with uninitialized state.
  \\
\hline
\hline
{\tt simplesine\_params.d}
  &
  Default \simplesine parameters.
  \\
\hline
\end{tabular}
\end{center}
}
