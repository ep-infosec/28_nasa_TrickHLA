\chapter{Federation Restore}
\label{sec:hla_trick_fed_restore}

A federation restore must occur in {\tt freeze} mode (see section~\ref{sec:interactive_restore} {\tt Interactive Restore} below)
or it must occur at simulation startup (see section~\ref{sec:prog_restore} {\tt Programmatic Restore}  below).

% ------------------------------------
\section{Interactive Restore }
\label{sec:interactive_restore}

Perhaps the most straightforward way to perform a federation restore is via the Trick Simulation Control Panel.
Simply click the {\tt Freeze} button on a federate's simulation control panel, and a freeze interaction is sent
so that all federates will freeze at the same time (usually one or two lookahead\_time frames after the freeze
click). Then click the {\tt Load Chkpnt} button to trigger the federation restore. A window will pop up where you 
can select an existing checkpoint file to be loaded.

See section~\ref{sec:hla_trick_fed_save} {\tt Federaton Save} above for how to dump a checkpoint file. 
Valid checkpoint file names to load will be of the form:

\begin{verbatim}
<federation_name>_<user_file_name>
\end{verbatim}

Each federate will load its own checkpoint file using the chosen file name.
Simply click the {\tt Start} button on the simulation control panel when you are ready to continue execution.

IMPORTANT: Use the same federate's Trick simulation control panel when clicking {\tt Freeze} and {\tt Start}.
Each federate may have its own control panel, but you must use the same control panel that you clicked {\tt Freeze}
on to then click {\tt Start} on. If you use a different control panel to click the {\tt Start} button, the simulation will most
likely not be able to continue.

% ------------------------------------
\section{Programmatic Restore }
\label{sec:prog_restore}

In the current {\tt TrickHLA} implementation, a programmatic federation restore can be initiated
ONLY from the first federate which creates the federation (the {\tt master}), and the restore will occur at simulation startup.
Since the only way to trigger a federation restore in this manner is via
the input file at the startup of the federation, you must provide
the name of the file to restore from in each input file and set the
{\tt restore\_federation} flag to true.

Note that only the {\tt master} federate needs to have the {\tt restore\_federation} flag
set in the input file, but setting it in every federate's input file is a way to ensure that the federation is restored at startup no matter
which federate is the {\tt master}.

The only other thing to set is the {\tt THLA.federate.HLA\_save\_directory}, 
which defaults to the {\tt RUN} directory if left unset.

The following is an example of triggering the federation restore at startup time via the Trick input file:

\begin{verbatim}
THLA.manager.restore_federation = 1;
THLA.manager.restore_file_name = "checkpoint.15.000";
\end{verbatim}

NOTE: You do not have to specify the federation name (prepended to the checkpoint file name) in the {\tt restore\_file\_name}.
{\tt TrickHLA} will automatically prepend the federation name to the file name you supply if need be. So in the above example, if the federation 
name is {\tt MyFederation}, then the file that will be loaded is {\tt MyFederation\_checkpoint.15.000} from the {\tt RUN} directory.
