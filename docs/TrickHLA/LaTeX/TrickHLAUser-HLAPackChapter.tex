\chapter{Data Encoding and Packing}
\label{sec:hla-pack}

\TrickHLA\ provides a mechanism for simulation developers to
modify updated class attribute data before sending it out to other federates
or upon receiving new data from publishers.
This capability might be necessary, for example, if the agreed-upon units
for an attribute are different from those used internally by the simulation,
in which case a developer could write {\em pack} and {\em unpack} logic
to change units as data come into and leaves the simulation.
This chapter illustrates how to do that.

% ---------------------------------------------------------------------------
\section{What is a {\em packing} class?}

\TrickHLA\ provides two {\em hooks} that allow simulation developers
to modify\footnote{
  What {\em modify} means is application specific.
  It might be encoding/decoding.
  It might be packing/unpacking.
  Or it might be changing units back and forth from FOM-agreed units
  (e.g., degrees) and application-specific units (e.g., radians).
}
data as it is sent out via HLA
and as it arrives from HLA.
This is implemented in the {\tt pack()} and {\tt unpack()}
methods of the {\tt TrickHLAPacking} class.
The {\tt pack()} method is automatically invoked by
\TrickHLA\ before data are sent via HLA.
The {\tt unpack()} method is automatically invoked by
\TrickHLA\ after data are received from HLA.
These are virtual methods and must be overriden in a subclass
in order to add application-specific packing and unpacking to the simulation.

% ------------------------------------
\subsection{{\tt TrickHLAPacking}}

The header file for the {\tt TrickHLAPacking} class is shown below.

\begin{lstlisting}[caption={{\tt TrickHLAPacking} class header}]
class TrickHLAObject;
#include "TrickHLA/include/TrickHLAObject.hh"

class TrickHLAPacking
{
   friend class InputProcessor;
   friend void init_attrTrickHLAPacking();

  public:
   TrickHLAPacking() {};
   virtual ~TrickHLAPacking() {};

   TrickHLADebugHandler debug_handler; // -- Prints out multiple debug levels

   TrickHLAAttribute * get_attribute( const char * attr_FOM_name );

   virtual void initialize_callback( TrickHLAObject * obj );

   //-----------------------------------------------------------------
   // These are virtual functions and must be defined by a full class.
   //-----------------------------------------------------------------

   virtual void pack();

   virtual void unpack();

  protected:
   TrickHLAObject * object;   // ** Reference to the TrickHLA Object.
};
\end{lstlisting}

% ------------------------------------
\subsection{{\tt simplesine\_Packing}}

In order to illustrate the use of the {\tt TrickHLAPacking} class,
the \simplesine model has a packing class.
The class header is shown below.

\begin{lstlisting}[caption={{\tt simplesine\_Packing} class header}]
#include "TrickHLA/include/TrickHLAPacking.hh"
#include "simplesine/include/simplesine.h"

class simplesine_Packing : public TrickHLAPacking
{
  friend class InputProcessor;
  friend void init_attrsimplesine_Packing();

  public:
    simplesine_Packing();
    virtual ~simplesine_Packing();

    virtual void init(
      simplesine_T* originalP,
      simplesine_T* packedP,
      simplesine_T* unpackedP );

    virtual void pack();
    virtual void unpack();

  private:
    bool is_initialized;
    simplesine_T* originalP;
    simplesine_T* packedP;
    simplesine_T* unpackedP;
};
\end{lstlisting}

This class is designed to work as follows.
the {\tt init()} method {\em must} be called in order to initialize the class
(i.e., invoked as a Trick initialization job).
The initialization method specifies three \simplesine objects:
\begin{itemize}
\item{{\tt originalP} --
  The \simplesine data used as input to the {\tt pack()} and {\tt unpack()}
  methods.
}
\item{{\tt packedP} --
  The \simplesine data used as output from the {\tt pack()} method.
}
\item{{\tt unpackedP} --
  The \simplesine data used as output from the {\tt unpack()} method.
}
\end{itemize}

The implementation of the methods is shown below.
In this case, both methods just copy the data without doing any
modification at all, which of course has no value other than
illustrating (in the following simulations) how to use the
{\tt TrickHLAPacking} class.

\begin{lstlisting}[caption={{\tt simplesine\_Packing} class methods}]
/********************************* TRICK HEADER *******************************
PURPOSE: (implementation of packing/unpacking methods)
LIBRARY DEPENDENCY: ((simplesine_copy.o))
*******************************************************************************/
// System include files.
#include <math.h>
#include <stdlib.h>
#include <iostream>
#include <string>

// Trick include files.
#include "sim_services/include/exec_proto.h"

// TrickHLA model include files.
#include "TrickHLA/include/TrickHLAAttribute.hh"

// Model include files.
#include "../include/simplesine_Packing.hh"
#include "../include/simplesine_proto.h"

using namespace std;

/*----------------------------------------------------------------------------
PURPOSE: (default constructor for the simplesine packing/unpacking class.)
-----------------------------------------------------------------------------*/
simplesine_Packing::simplesine_Packing() // RETURN: -- None.
  : is_initialized(false),
    originalP(NULL),
    packedP(NULL),
    unpackedP(NULL)
{ }

/*----------------------------------------------------------------------------
PURPOSE: (destructor for the simplesine packing/unpacking class.)
-----------------------------------------------------------------------------*/
simplesine_Packing::~simplesine_Packing() // RETURN: -- None.
{ }

/*----------------------------------------------------------------------------
PURPOSE: (initialization method)
-----------------------------------------------------------------------------*/
void simplesine_Packing::init( // RETURN: -- None.
  simplesine_T* originalP,     // INOUT:  -- where the data is coming from
  simplesine_T* packedP,       // INOUT:  -- where to pack it into
  simplesine_T* unpackedP)     // INOUT:  -- where to unpack it into
{
  this->originalP = originalP;
  this->packedP = packedP;
  this->unpackedP = unpackedP;
  this->is_initialized = true;
}

/*----------------------------------------------------------------------------
PURPOSE: (data packing method. This is called before data is sent to the RTI.)
-----------------------------------------------------------------------------*/
void simplesine_Packing::pack()  // RETURN: -- None.
{
  if( ! this->is_initialized ) {
    exec_terminate(
      "simplesine_Packing.cpp",
      "simplesine_Packing::pack() called on non-initialized object" );
  }

  send_hs( stdout, "pack(): packing data from %p into %p", originalP, packedP );
  simplesine_copy( originalP, packedP );
}

/*----------------------------------------------------------------------------
PURPOSE: (data unpacking method. This is called after data is received
  from the RTI.)
-----------------------------------------------------------------------------*/
void simplesine_Packing::unpack() // RETURN: -- None.
{
  if( ! this->is_initialized ) {
    exec_terminate(
      "simplesine_Packing.cpp",
      "simplesine_Packing::unpack() called on non-initialized object" );
  }

  send_hs( stdout, "unpack(): unpacking data from %p into %p", originalP, unpackedP );
  simplesine_copy( originalP, unpackedP );
}
\end{lstlisting}

% ---------------------------------------------------------------------------
\section{\tt SIM\_simplesine\_hla\_pack}

This {\tt SIM\_simplesine\_hla\_pack} simulation illustrates how to
{\em pack} data from a publishing simulation.
The simulation is based on the plain publisher,
{\tt SIM\_simplesine\_hla\_pub}, from
Chapter~\ref{sec:hla-pubsub}.

% -----------------------
\subsection{\sdefine file}
This differences between the \sdefine for this simulation and that
of the plain publisher are as follows.

\begin{itemize}
\item{
  A new \simplesine variable, {\tt simplesine\_packed},
  is defined in the {\tt publisher} sim object.
}
\item{
  The definition of an instance, {\tt packer},
  of the {\tt simplesine\_Packing} class.
}
\item{
  The invocation of {\tt simplesine\_Packing::init()} as a Trick
  initialization job,
  specifying {\tt publisher.simplesine} as the input to the {\tt pack()}
  method and {\tt publisher.simple\-sine\_\-packed} as the output.
}
\end{itemize}

% -----------------------
\subsection{Input file}

The only difference between the input file for this simulation and
that of the plain publisher is the following new line

\begin{verbatim}
THLA.manager.objects[0].packing = &publisher.packer; \end{verbatim}

which associates {\tt publisher.packer} as the packing class for
the object being published.

% ---------------------------------------------------------------------------
\section{\tt SIM\_simplesine\_hla\_unpack}

This {\tt SIM\_simplesine\_hla\_unpack} simulation illustrates how to
{\em unpack} data from a subscribing simulation.
The simulation is based on the plain subscriber,
{\tt SIM\_simplesine\_hla\_sub}, from
Chapter~\ref{sec:hla-pubsub}.

% -----------------------
\subsection{\sdefine file}
This differences between the \sdefine for this simulation and that
of the plain subscriber are as follows.

\begin{itemize}
\item{
  A new \simplesine variable, {\tt simplesine\_unpacked},
  is defined in the {\tt subscriber} sim object.
}
\item{
  The definition of an instance, {\tt unpacker},
  of the {\tt simplesine\_Packing} class.
}
\item{
  The invocation of {\tt simplesine\_Packing::init()} as a Trick
  initialization job,
  specifying {\tt subscriber.simplesine} as the input to the {\tt unpack()}
  method and {\tt subscriber.simple\-sine\_\-un\-packed} as the output.
}
\end{itemize}

% -----------------------
\subsection{Input file}

The only difference between the input file for this simulation and
that of the plain subscriber is the following new line

\begin{verbatim}
THLA.manager.objects[0].packing = &subscriber.unpacker; \end{verbatim}

which associates {\tt subscriber.unpacker} as the packing class for
the object to which the simulation subscribes.

% ---------------------------------------------------------------------------
\section{Output}

The output of running the pack and unpack simulations
shows issued by the \simplesine {\tt pack()} and
{\tt unpack()} methods.

\begin{lstlisting}[numbers=none,caption={Output from the packer simulation}]
...
| |wormhole|1|0.00|2007/08/05,23:39:14| TrickHLAManager::initialization_complete()
        Simulation has started and is now running...
| |wormhole|1|0.00|2007/08/05,23:39:14| TrickHLAManager::receive_cyclic_data()
| |wormhole|1|0.00|2007/08/05,23:39:14| TrickHLAManager::send_cyclic_and_requested_data()
| |wormhole|1|0.00|2007/08/05,23:39:14| pack(): packing data from 0x99aaee4 into 0x99aaf1c
| |wormhole|-1|0.25|2007/08/05,23:39:14| TrickHLAFedAmb::timeAdvanceGrant()
Federate "publisher" Time granted to: 1
| |wormhole|1|1.00|2007/08/05,23:39:15| TrickHLAManager::receive_cyclic_data()
| |wormhole|1|1.00|2007/08/05,23:39:15| TrickHLAManager::send_cyclic_and_requested_data()
| |wormhole|1|1.00|2007/08/05,23:39:15| pack(): packing data from 0x99aaee4 into 0x99aaf1c
| |wormhole|-1|1.25|2007/08/05,23:39:15| TrickHLAFedAmb::timeAdvanceGrant()
Federate "publisher" Time granted to: 2
| |wormhole|1|2.00|2007/08/05,23:39:16| TrickHLAManager::receive_cyclic_data()
| |wormhole|1|2.00|2007/08/05,23:39:16| TrickHLAManager::send_cyclic_and_requested_data()
| |wormhole|1|2.00|2007/08/05,23:39:16| pack(): packing data from 0x99aaee4 into 0x99aaf1c
| |wormhole|-1|2.25|2007/08/05,23:39:16| TrickHLAFedAmb::timeAdvanceGrant()
Federate "publisher" Time granted to: 3
| |wormhole|1|3.00|2007/08/05,23:39:17| TrickHLAManager::receive_cyclic_data()
| |wormhole|1|3.00|2007/08/05,23:39:17| TrickHLAManager::send_cyclic_and_requested_data()
| |wormhole|1|3.00|2007/08/05,23:39:17| pack(): packing data from 0x99aaee4 into 0x99aaf1c
| |wormhole|-1|3.25|2007/08/05,23:39:17| TrickHLAFedAmb::timeAdvanceGrant()
Federate "publisher" Time granted to: 4
...
\end{lstlisting}

\begin{lstlisting}[numbers=none,caption={Output from the unpacker simulation}]
...
| |wormhole|1|0.00|2007/08/05,23:39:14| TrickHLAManager::initialization_complete()
        Simulation has started and is now running...
| |wormhole|1|0.00|2007/08/05,23:39:14| TrickHLAManager::receive_cyclic_data()
| |wormhole|1|0.00|2007/08/05,23:39:14| TrickHLAManager::send_cyclic_and_requested_data()
| |wormhole|1|0.00|2007/08/05,23:39:14| pack(): packing data from 0x99aaee4 into 0x99aaf1c
| |wormhole|-1|0.25|2007/08/05,23:39:14| TrickHLAFedAmb::timeAdvanceGrant()
Federate "publisher" Time granted to: 1
| |wormhole|1|1.00|2007/08/05,23:39:15| TrickHLAManager::receive_cyclic_data()
| |wormhole|1|1.00|2007/08/05,23:39:15| TrickHLAManager::send_cyclic_and_requested_data()
| |wormhole|1|1.00|2007/08/05,23:39:15| pack(): packing data from 0x99aaee4 into 0x99aaf1c
| |wormhole|-1|1.25|2007/08/05,23:39:15| TrickHLAFedAmb::timeAdvanceGrant()
Federate "publisher" Time granted to: 2
| |wormhole|1|2.00|2007/08/05,23:39:16| TrickHLAManager::receive_cyclic_data()
| |wormhole|1|2.00|2007/08/05,23:39:16| TrickHLAManager::send_cyclic_and_requested_data()
| |wormhole|1|2.00|2007/08/05,23:39:16| pack(): packing data from 0x99aaee4 into 0x99aaf1c
| |wormhole|-1|2.25|2007/08/05,23:39:16| TrickHLAFedAmb::timeAdvanceGrant()
Federate "publisher" Time granted to: 3
| |wormhole|1|3.00|2007/08/05,23:39:17| TrickHLAManager::receive_cyclic_data()
| |wormhole|1|3.00|2007/08/05,23:39:17| TrickHLAManager::send_cyclic_and_requested_data()
| |wormhole|1|3.00|2007/08/05,23:39:17| pack(): packing data from 0x99aaee4 into 0x99aaf1c
| |wormhole|-1|3.25|2007/08/05,23:39:17| TrickHLAFedAmb::timeAdvanceGrant()
Federate "publisher" Time granted to: 4
...
\end{lstlisting}
